\documentclass[10pt]{article}
\usepackage[usenames]{color} %used for font color
\usepackage{amssymb} %maths
\usepackage{amsmath} %maths
\usepackage[utf8]{inputenc} %useful to type directly diacritic characters
\begin{document}
Veronika Nechaeva | CSE015 Homework 3 | Basic Counting Principles
 \\[\baselineskip]
1. To solve this problem, I will use the permutations formula. 
$P(n,r)=n!/(n-r)!$ We want to see how many arrangements of the 
English alphabet there are. English alphabet contains 26 letters.
To see how many different ways there are for 26 letters to be arranged, 
we can plug 26 for n and r. So, $P(26,26)=26!/(26-26)!$.
This evaluates to $P(26,26)=26!$, which makes sense, because there are 
26 spots. For the first spot, there are 26 different possible letters
that can fill it. For the second spot, there is now 25, because one
of the 26 letters already took the first spot. For the third spot 
there is 24 letters to choose from, for the fourth - 23. And so on.
This can be written as 26! arrangements. \\[\baselineskip]

2. There are 18 math majors and 325 computer science majors. When picking two representatives 
where one of them has math major, and the other one has a computer science major, there is 2 possible
outcomes: 1st representative-math major, second-computer science major, 
and vice versa. There are 18 ways for 1 representative to get picked from 
the math major group and there are 325 ways for 1 representative to get picked 
from the computer science major group. From this we get that the number of ways for 
two representatives can be picked so that one is a mathematics major and the other is a 
computer science major equals 2*18*325 = 11,700 \\[\baselineskip]

3. There are 12 colors of the shirt, 2 models of the shirt (for males and 

females), and 3 sizes for each model of the shirt. So there are 

12*3 shirt types for women (12 colors and 3 sizes), and 12*3 shirt types

for men. In total there are 12*3*2=72 different shirt types. \\[\baselineskip]

4. There are 10 questions and 4 answer choices for each. The number 

of ways a student can answer this quiz is equal to 4*4*4...(10 times).

So, it equals to $4^{10}=1,048,576$. \\[\baselineskip]

5. We know how many ways a student can answer a quiz of 10 questions if

we assume he answers them all from number 4. It is $4^{10}$. Now we need to

calculate all the possible ways for the student to answer a test with 1 

question left blank, then 2 questions left blank, then 3 and 4, all the way

to 10 questions left blank, and sum all results. When 1 question is left 

blank, we can treat the quiz like it has 9 questions. So, there are $4^9$ ways

to answer the quiz with 1 blank answer. Same logic applies for the rest of 

the scenarios. 

In total, there are $4^{10}+4^9+4^8+4^7+4^6+4^5+4^4+4^3+4^2+4^1+4^0=$

1,398,101 ways to answer a 10 question quiz if you can leave answers blank. 
\end{document}