\documentclass[10pt]{article}
\usepackage[usenames]{color} %used for font color
\usepackage{amssymb} %maths
\usepackage{amsmath} %maths
\usepackage[utf8]{inputenc} %useful to type directly diacritic characters
\begin{document}
3. p $\rightarrow$ q  and $\neg$q $\rightarrow$ $\neg$p
\begin{displaymath}
\begin{array}{|c c|c|c|c|c|}
p & q & p \rightarrow q & \neg q & \neg p & \neg q \rightarrow \neg p\\
\hline
T & T & T & F & F & T\\
T & F & F & T & F & F\\
F & T & T & F & T & T\\
F & F & T & T & T & T\\
\end{array}
\end{displaymath}

As we can see, for all 4 cases these two expressions have the same results. 

Therefore, they are equivalent.

4. p $\rightarrow$ q  and $\neg$p $\lor$ q
\begin{displaymath}
\begin{array}{|c c|c|c|c|}
p & q & p \rightarrow q & \neg p & \neg p \lor q\\
\hline
T & T & T & F & T\\
T & F & F & F & F\\
F & T & T & T & T\\
F & F & T & T & T\\
\end{array}
\end{displaymath}

As we can see, for all 4 cases these two expressions have the same results. 

Therefore, they are equivalent.

5. $\neg$ (p $\land$ q) and $\neg$p $\lor$ $\neg$q
\begin{displaymath}
\begin{array}{|c c|c|c|c|c|c|}
p & q & \neg p & \neg q & p \land q & \neg (p \land q) & \neg p \lor \neg q\\
\hline
T & T & F & F & T & F & F\\
T & F & F & T & F & T & T\\
F & T & T & F & F & T & T\\
F & F & T & T & F & T & T\\
\end{array}
\end{displaymath}

As we can see, for all 4 cases these two expressions have the same results. 

Therefore, they are equivalent.


\end{document}