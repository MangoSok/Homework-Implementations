\documentclass[11pt]{article}

% To produce a letter size output. Otherwise will be A4 size.
\usepackage[letterpaper]{geometry}

% For enumerated lists using letters: a. b. etc.
\usepackage{enumitem}

\topmargin -.5in
\textheight 9in
\oddsidemargin -.25in
\evensidemargin -.25in
\textwidth 7in

\begin{document}
% Edit the following putting your first and last names and replace XXX with your lab section (e.g., 03L).
\author{Veronika Nechaeva\\
Lab S21-CSE 015 08L}

% Edit the following replacing X with the HW number.
\title{CSE 015: Discrete Mathematics\\
Spring 2021\\
Final Homework\\
Solution}

% Put today's date in the following.
\date{May 14, 2021}
\maketitle

\section{Logic and Proofs}

\subsection{Propositional Logic}

\begin{enumerate}[label=(\alph*)]

\item 
%You provide the solution to question 1.1(a) here.
$\neg p\land (p\lor\neg q)$ Translation: Freeway is closed and no accidents happen.

\item 
%You provide the solution to question 1.1(b) here.
$\neg p\rightarrow\neg q$ Translation: If freeway is closed, then No accidents will happen.

\end{enumerate}


\subsection{Logical Equivalences}

%Your answer

Let A = $p\rightarrow q$ and B = $\neg r \rightarrow \neg q$
\begin{displaymath}
\begin{array}{|c c|c c|c c|c c|c c|c|c|c|c|}
p & q & r & q & r & p & \neg r & \neg q & \neg r & \neg p & A & B & A \lor B & \neg r \rightarrow \neg p\\
\hline
T & T & T & T & T & T & F & F & F & F & T & T & T & T\\
T & F & T & F & T & F & F & T & F & T & F & T & T & T\\
F & T & F & T & F & T & T & F & T & F & T & F & T & F\\
F & F & F & F & F & F & T & T & T & T & T & T & T & T\\
\end{array}
\end{displaymath}

As you can see, $A\lor B$ and $\neg r \rightarrow \neg p$ are NOT equivalent. 


\subsection{System Specifications}
%Your answer
A = system is in active state 

B = system is operating normally

C = user action is legal 

D = system is in interrupt mode


\subsection{Quantifiers}

\begin{enumerate}[label=(\alph*)]

\item
%You provide the solution to question 1.4(a) here.
$\forall x(x^2-1>0)$ This statement is false because 0 is a real number, and $0^2-1=-1$, which is less than 0. Also, when x is 1, the result will equal to 0. Therefore, it is a false statement.

\item
%You provide the solution to question 1.4(b) here.
$\exists x \exists y(x^4+y^4=0)$ This statement is true since there exists at least one x and y value which satisfies this expression. That is x = 0 and y = 0 (because 0 is a real number and $0^4+0^4=0$). 

\end{enumerate}


\subsection{Negation of Complex Sentences}

\begin{enumerate}[label=(\alph*)]

\item
%You provide the solution to question 1.5(a) here.

\item
%You provide the solution to question 1.5(b) here.

\end{enumerate}

\section{Counting}

\begin{enumerate}[label=(\alph*)]

\item
%You provide the solution to question 2(a) here.
I solved this problem in two ways: using logic and using code. 

Logic Solution: Our range is from 1 to 99(inclusive). So there are 99 numbers. 

Approximately half of those numbers is odd. Since we start and end with an odd number

(1 and 99), the number of odd numbers is $int(99/2)+1=50$. Now we need to find all the

positive integers in out range that are perfect squares. We need to remember that we 

already included some of them in the 50 that we got in the calculation above. 

So we need to ignore the perfect squares that are also odd. We will only look at even 

perfect squares in our range. To find how many of them there are, we will look at squares 

of all the even positive integers below 10 (since $10^2=100$, which is out of our range).

There are 4 positive integers below 10 that produce even squares: 2, 4, 6, 8. Now we 

just sum up the two numbers we got: $50+4=54$. There are 54 positive integers below

100 that are either odd or are perfect squares.

Code Solution: I implemented an algorithm which has a for loop from i=1 to 100 (exclusive), 

and each iteration checks wether i is odd or a square of an integer. If i is one of

those, then it increments count by 1.

Code:

import math

def countNum(num1,num2):
    count=0
    for i in range(num1,num2):
        if $i\%2$ == 1 or int(math.sqrt(i) + 0.5) ** 2 == i:
            count += 1
    return count
    
print(countNum(1,100))


\item
%You provide the solution to question 2(b) here.

Let C be calculator percentage, $C=98\%$, and S be the smartphone percentage, $S=96\%$.

Since there is $95\%$ of students that have both a smartphone and a calculator, we

can find what percentage of students only have a phone or only have a calculator.

Only have a phone: $96-95=1\%$. Only have a calculator: $98-95=3\%$. 

So the percentage of students that do not have either a phone or a calculator is

$100-95-3-1=1\%$. Only $1\%$ of students have neither a smartphone or a calculator.

\end{enumerate}



\section{Sets}

\begin{enumerate}[label=(\alph*)]

\item
%You provide the solution to question 3(a) here.
$A\cup B\cup C={0,1,2,3,4,5,6,7,8,9,10}$

\item
%You provide the solution to question 3(b) here.
$(A\cup B)\cap C$ Solution: Let D be the set of $(A\cup B)$. $D={0,1,2,3,4,5,6,8,10}$

So $D\cap C={4,5,6,8,10}$
\end{enumerate}



\section{Functions}

\begin{enumerate}[label=(\alph*)]

\item
%You provide the solution to question 4(a) here.
$f(x,y)=x^2-y$ is surjective, where x,y,z are all integers. It is surjective because for 

all integer output values there is at least one input x and y value. Let's say $f(x,y)=a$ 

and $x=b$, where a and b are any integer. We get $a=b^2-y$, so for any a and b value,

y will be $y=b^2-a$. 

\item
%You provide the solution to question 4(b) here.
$f(x,y)=x-y$ is onto, where x,y,z are all integers. It is onto since for every possible

integer output there is at least one x and y input value that would result in that output.

Formula: $z=x-y$, so for any possible z and x integer value, y would be $y=x-z$. 

\item
%You provide the solution to question 4(c) here.
$f(s)=(x^2+1)/(x^2+2)$ this function is not a bijection because it is definitely not a 

surjection. That function can never give a negative output since the input is squared 

(and in increased by 1 or 2) and input is restricted to real numbers. In order for a 

function to be bijective, it needs to be both onto and 1-to-1. Since this function is

not onto, it is not bijective.

\end{enumerate}



\section{Sequences and Sums}

\begin{enumerate}[label=(\alph*)]

\item
%You provide the solution to question 5(a) here.
$a_0=2*(-3)^0+5^0=2*1+1=3$ 

\item
%You provide the solution to question 5(b) here.
$a_n=n/2$ so if we plug in 2, we get $a_2=2/2=1$
\end{enumerate}



\section{Algorithms}

\begin{enumerate}[label=(\alph*)]

\item
%You provide the solution to question 6(a) here.
Pseudocode: 

temp = m

m = n

n = temp

\item
%You provide the solution to question 6(b) here.
Assuming that key() function is O(1) complexity, big-O estimate for this algorithm is 

O(max(length(input),k)) since we dont know what is bigger, length(count) or length(input). 

However, if we assume that k is a constant, our complexity becomes O(n) where n is the

length of input. 

\item
%You provide the solution to question 6(с) here.
We can make 3 arrangements, each made of 2 functions, in a way that first function

is the big-O of the next one. First: $1.5^n$, $(log_2n)^3$. Second: $10^n$, $n^{99}$.

Third: $n!$, $n^{100}$. In each of the arrangements, the first function is O(n)

and the second function is O(1).



\end{enumerate}



\section{Number Theory}

\begin{enumerate}[label=(\alph*)]

\item
%You provide the solution to question 7(a) here.
$5+_m8\rightarrow (5+8)mod13=0$. So $5+_m8=0$ when $m=13$.

\item
%You provide the solution to question 7(b) here.
$66/7 \rightarrow$ Quotient = 9. Remainder = 3. $3/7$ Quotient = 0. Remainder = 3. So $66\equiv3(mod7)$

\item
%You provide the solution to question 7(b) here.
$80+11=91$. $91mod12=7$. Quotient = Remainder = 7. The clock shows 7:00.

\end{enumerate}

\end{document}