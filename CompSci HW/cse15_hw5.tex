\documentclass[10pt]{article}
\usepackage[usenames]{color} %used for font color
\usepackage{amssymb} %maths
\usepackage{amsmath} %maths
\usepackage[utf8]{inputenc} %useful to type directly diacritic characters
\begin{document}
 Veronika Nechaeva

Discrete Mathematics - Homework 5 / Algorithms $\newline$

1. An algorithm that takes as input a list of n integers and finds the location 

of the last odd integer in the list or returns -1 if there are no odd integers in

the list:

This function takes in an integer list n

Create a variable num holding a value -1

Iterate over each element in list n

If element mod 2 equals 1 (meaning it's odd), assign num to that element

When done iterating over elements, return num $\newline$

2.  An algorithm that inserts an integer a in the appropriate position into 

the list x1, x2, ..., xn of integers that are in decreasing order:

This function takes in an integer list n (decreasing) and an integer num.

If n is empty or last element in n is bigger than num, simply append num.

Else if the first element in n is smaller than num, insert num at index 0.

Else , iterate over each element in n by using index integer i.

If element at position i is bigger than num and element at position 

i + 1 is smaller than num, insert num at position i + 1. $\newline$

Order of Complexity

1. a. $f(x)=17x+11$ is a $O(x^2)$. Witnesses: C = 2 and k = 9 

(9.104 to be exact). So we have $17x+11\leq2x^2$ when $x>9$.

b. $f(x)=x\log_2x$ is a $O(x^2)$. Witnesses: C = 2 and k = 0.

So we have $x\log_2x\leq2x^2$ when $x>0$.

c. $f(x)=x^4/2$ is Not a $O(x^2)$ $\newline$

2. In the algorithm provided, there is two for loops that go from 0 to n. 

Which means that if n = 4, the inner loop will repeat 16 times, so at the

end t will be equal to 16. 16 is a square of 4, therefore the big-O

estimate for this algorithm is $O(n^2)$ $\newline$

3. a. $f(n)=\log_2n$ We get that $\log_2n\leq10^{-9}$ 

So largest n would be $2^{10^{-9}}$

b. $f(n)=n$ We get that $n\leq10^{-9}$ So largest n would be $10^{-9}$

c. $f(n)=2^n$ We get that $2^n\leq10^{-9}$ So largest n would be $\log_210^{-9}$

d. $f(n)=n!$ We get that $n!\leq10^{-9}$. Factorial function is not defined 

for such small decimal numbers, so there is no such n besides zero.

\end{document}