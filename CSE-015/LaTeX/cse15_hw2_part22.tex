\documentclass[10pt]{article}
\usepackage[usenames]{color} %used for font color
\usepackage{amssymb} %maths
\usepackage{amsmath} %maths
\usepackage[utf8]{inputenc} %useful to type directly diacritic characters
\begin{document}
Nested Quantifiers
\\[\baselineskip] 

1. $\exists$x $\forall$y A(x,y)   
Value: True 

Reasoning: when x = 0, x times any y will equal 0.

2. $\exists$x $\exists$y B(x,y)   
Value: True 

Reasoning: there is multiple values for x and y that 

result in x+y = 0. For example, x = -1 and y = 1.

3. $\forall$x $\exists$y A(x,y)    
Value: True

Reasoning: when y = 0, y times any x will equal 0.

4. $\exists$x $\forall$y (A(x,y) $\land$ B(x,y))
Value: False

Reasoning: only 1 x value makes x*y=0 always true, x = 0.

And if you plug in x = 0 to B(x,y) , you can see that 

there is only 1 y value that will result in x + y = 0,

that is y = 0. But for the rest on the y values, 

the result will not be 0. Therefore, there does not exist 

a value x for all y that would satisfy both A(x,y) and B(x,y).

5. $\exists$x $\exists$y (A(x,y) $\land$ $\neg$ B(x,y))
Value: True

Reasoning: For example, take x = 0 and y = 5. If you plug them

in A(x,y), you get 0*5 = 0. However, if you plug them in B(x,y),

you get 0 + 5 = 0, which is not true. So there exists at least one 

value for x and for y that results in A(x,y) being true and 

$\neg$ B(x,y) being false.
\end{document}