\documentclass[10pt]{article}
\usepackage[usenames]{color} %used for font color
\usepackage{amssymb} %maths
\usepackage{amsmath} %maths
\usepackage[utf8]{inputenc} %useful to type directly diacritic characters
\begin{document}
Veronika Nechaeva | CSE015 - Homework 3 | Mathematical Proofs

1. The square of an even number is even.

Let n be an even number. n can be represented as n = 2k (as any other even 

number), where k is an integer. $n^2=(2k)^2=4*k^2=2(2k^2)$. So, 

$n^2=2*(2k^2)$, where $(2k^2)$ is an integer, since k is an integer. 

Since $n^2$ can take a form of 2*integer, it must be an even number.  \\[\baselineskip]

2. The product of two odd integers is odd.

Let n and m be odd integers. Since they are both odd integers, they can 

take on a form $n=2k+1$ and $m=2x+1$, where k and x are also integers. 

Therefore, $n*m=(2k+1)(2x+1)=4kx+2k+2x+1$. Since multiplying 

any integer by an even integer results in an even integer and summing 

all even numbers gives you an even number, $4kx+2k+2x$ gives an even 

integer. Adding one to that even integer results in an odd integer. 

Therefore, the product of two odd integers must result in an odd integer.  \\[\baselineskip]

3. If $n^3+5$ is odd then $n$ is even, for any $n\in Z$ 

Since 5 is an odd integer, adding it to the odd number will result in an even 

number, but adding it to an even number will result in an odd number. 

In this problem, the expression $n^3+5$ is assumed to be odd. Therefore, 

$n^3$ must be even to result for $n^3+5$ to be odd. 

Let's assume $n$ is odd. As proven in the previous problem, any odd 

integer times another odd integer will result in an odd integer. 

If $n$ is odd, then $n^3=n*n*n$ is also odd, because $n*n$ will give 

an odd number, and multiplying that by $n$ again is also an odd number. 

Therefore, in $n^3+5$ , $n$ cannot be odd so that the whole expression 

evaluates to an odd number, so $n$ must be even. \\[\baselineskip]

4. If $3n+2$ is even then $n$ is even for any $n\in Z$ 

A sum of 2 even numbers will result in an even number, while the sum of 

one odd number and one even number will result in an odd number. 

Therefore, $3n$ must be even for $3n+2$ to be even. Let's assume $n$ is odd. 

As we proved in problem 2, multiplying any odd integer by an odd integer 

results in an odd integer. 3 is an odd integer, $n$ is too. Therefore, 

$3n+2$ will also be odd. So, $n$ must be even in order for the

expression $3n+2$ to be even. \\[\baselineskip]

5. The sum of a rational number and an irrational number is irrational.

Let k be a rational number and i be an irrational number. Let's assume

$k+i$ evaluates to a rational number n. If $k+i=n$, then $i=n-k$.

However, $i$ is an irrational number, but $n-k$ is a rational number.

Therefore, $i$ cannot equal to $n-k$. So n must also be irrational. 


\end{document}