\documentclass[10pt]{article}
\usepackage[usenames]{color} %used for font color
\usepackage{amssymb} %maths
\usepackage{amsmath} %maths
\usepackage[utf8]{inputenc} %useful to type directly diacritic characters
\begin{document}
Veronika Nechaeva, Spring 2021

CSE-015: Discrete Mathematics

Homework 4 / Functions 
$\medskip$

1. A function is considered to be R to R if f is a function from the real 

numbers to the real numbers. In other words, if the input of the function 

is a real number, the function will output a real number.

a) $f(x)=\pm\sqrt{x^2+1}$  This function is R to R because the value inside the 

square root will always be positive (if the input is real) since $x$ (the input)

 is squared.

b) $f(x)=1/x$ This function is NOT R to R because if the input is zero, 

which is a real number, the output is $1/0$, which is undefined. 

Undefined is neither real or imaginary.

c) $f(x)=x-x^2$ This function is R to R since when you plug any real 

number, you just square it (which also results in a real number) and

subtract it from itself. So the result will always be real if the inout is real.
$\medskip$
2. Finding Domain and Range for functions:

a) The function that assigns to each positive integer the largest perfect square 

not exceeding this integer. Domain: set of positive integers (Z+). Range: 

set of perfect square values. 

b) The function that assigns to each bit string the number of ones in the 

string minus the number of zeros in the string. Domain: set of bit 

strings (of any length). Range: all integers (Z).

c) The function that assigns to each bit string twice the number of zeros 

in that string. Domain: all bit strings. Range: all even integers

(zero included).

d) The function that assigns the number of bits left over when a bit string 

is split into bytes (which are blocks of 8 bits). Domain: set of bit 

strings (of any length). Range: integers in the interval from 0 to 8 

(zero included, eight excluded).

e) This part answered itself so there is no need to repeat it.
$\medskip$

3. Determine whether the function $f:\mathbb{Z}\times\mathbb{Z}\to\mathbb{Z}$

a) $f(m,n)=m^2-n^2$  this function is not onto because there is no 

integer values for m and n(in Z) which will result in $f=2$. In order

for a function to be onto, there should be m and n integer value 

for any f=c (where c is an integer in Z). In this case, $f$ can only 

equal to 1 or an integer greater than 2.

(since our condition is $f:\mathbb{Z}\times\mathbb{Z}\to\mathbb{Z}$).

b) $f(m,n)=m+n+1$ this function is onto because for any integer c there 

will be an integer m and n such that $f(m,n)=c$.

c) $f(m,n)=|m|-|n|$ this function is onto because for any integer c there 

will be an integer m and n such that $f(m,n)=c$.

d) $f(m,n)=m^2-4$ this function is not onto because there is no 

integer values for m and n(in Z) which will result $f=1$. In this case,

$f$ can only equal to -4,-3,0,5... As you can see, there is no 1.




\end{document}